%(Foster intro: 2660 words)
%(Matthews intro: 4855 words)

\section{A Brief History of Early Molecular Cloud Research}

In the mid 18th century, scientists like Emanuael Swedenborg, Immanuel Kant, and Pierre Laplace anticipated the importance of an interstellar medium (ISM) of gas. They hypothesized (mostly correctly) that the ISM provided the raw materials for stars like our sun, and also provided -- in the form of an accretion disk -- a sink for protostars to shed their excess angular momentum as they collapse. The intuition of these early scientists is especially impressive considering that the only observational support for this nebular hypothesis were the poorly-understood handful of naked-eye nebulae like the Orion Nebula. The first convincing measurement of the interstellar medium came from \cite{Hartmann04}, who observed Calcium absorption lines in a spectroscopic binary. Because the calcium lines did not share the doppler motion of the stars, Hartmann realized that the absorbing material was unassociated with the stellar atmosphere, and instead in a foreground of diffuse material.

Early observational studies of the interstellar medium relied on measuring the absorption of background starlight \citep{Barnard19, Bates51}. The modern observational study of molecular clouds began with the radio detection of molecular \textit{emission} from clouds \citep{Gundermann65, Palmer67, Wilson70}. Radio observations have the advantage that they measure the velocity-induced doppler-shift of cloud radiation, and hence provide hints about cloud motion. \cite{Goldreich74} commented on this potential:

\blockquote{Perhaps even more exciting is the knowledge that [radio observations] provide concerning the dynamics of gas clouds in the act of star formation. Although spectacular observational results have already been obtained, the theoretical attempts to interpret the existing data have been rather limited and somewhat inconclusive.}

This quote is remarkably prescient about the subsequent four decades of research. Properly interpreting the morphological clues from molecular cloud observations has been frustratingly difficult, despite continued dramatic improvements in the quality of our data. Indeed, when \cite{Goldreich74} attempted to interpret the dynamical clues of their radio data, they came to the conclusion that molecular clouds are in a state of free-fall collapse. This problematic interpretation has largely been dismissed, as it suggests that cloud collapse should quickly form stars at a rate 100 times higher than the observed star formation rate. \cite{Larson81} laid out the now-canonical interpretation of cloud dynamics -- that these large motions are signatures of supersonic turbulence, which support clouds against rapid gravitational collapse. The question of what sustains the turbulence in clouds remains open, and some researchers still posit that clouds are in free-fall collapse \citep{Vaz07}.

What makes the physical interpretation of molecular cloud data so difficult? Two factors are important. First, molecular clouds are dominated by turbulence, which remains one of the most poorly understood branches of physics. Indeed, nobody has even been able to prove that the Navier-Stokes equations -- which describe turbulent fluid mechanics -- are always solvable.\footnote{This is one of seven famous ``Millennium Prize Problems'' in mathematics, and it's solution will earn you \$1 million from the Clay Mathematics Institute. If somebody proves that the Navier Stokes Equations do \emph{not} always have a solution, that would almost certainly suggest that they cannot be a valid description of fluid mechanics.} The intractability of turbulent fluid mechanics makes it difficult to generate simple observational predictions, or to assess whether or not a particular observation is consistent with a particular physical model.

The second major difficulty in studying molecular clouds derives from their structural complexity -- clouds are highly structured, and always viewed in projection. Thus, compared to geometrically simpler objects like stars, it is more difficult to determine how projected measurements of clouds relate to three-dimensional cloud properties.

\section{The Need for Better Analysis}

Since the work of \cite{Goldreich74}, dramatic progress has been made in observing and simulating molecular clouds. However, the quote above still reflects the pesistent difficulty about how to interpret molecular cloud data. Many fundamental questions about molecular clouds remain unsettled. Among these unresolved questions are: is a typical cloud gravitationally bound? What are the most important mechanisms to inject turbulence into molecular clouds? To what extent is star formation in molecular clouds triggered? What determines the star formation efficiency of a molecular cloud? The difficulty of these questions derives from the inherent difficulty in analyzing observations of molecular clouds. Molecular clouds are among the most structurally complex objects in astrophysics, and these structures are our primary clues about the underlying astrophysics. This demands the use of sophisticated techniques to characterize and interpret the morphological information in molecular cloud data. Unlike the dramatic improvement in observation and simulation technology over the past 40 years, progress in analysis has been slower.

The majority of molecular cloud research relies on two types of measurement techniques. The first is a localized approach that segments a cloud into substructures, and extracts summary properties of each substructure -- quantities such as size, velocity dispersion, mass, virial parameter, luminosity and pressure. The major challenge of the localized approach is to decide how to meaningfully identify cloud substructures in the presence of projection effects and without sharp substructure boundaries. Early ``blob-segmentation'' approaches decomposed clouds into roughly spherical substructures \citep{Stutzki90, Williams94}. However, as the spatial dynamic range of observations improved, these simple segmentations became increasingly crude approximations of filamentary and hierarchical clouds. \cite{Pineda09} demonstrated that naively applying these blob-segmentation algorithms to hierarchical datasets introduces significant bias in subsequent measurements. New explicitly hierarchical segmentations \citep{Rosolowsky08} better describe the structural complexity of molecular clouds. Still, these techniques assume that isosurface contours in position-position (PP) or PPV datasets demarcate meaningfully different regions in the physical cloud. This is a strong assumption, and one I explore in some detail in Chapter 5.

The second, global approach to cloud measurement avoids identifying specific substructures. Examples of global measurements include the power spectrum and structure functions, which measure the relative presence of large- and small-scale intensity correlations in a cloud, and distribution functions of quantities like the column density. These particular global summaries have the advantage that theories make predictions about their functional form. A simple energy conservation argument suggests that incompressible turbulence generates a power spectrum of $E(k) \propto k^{-\frac{5}{3}}$  \citep{Elmegreen04}. Likewise, log-normal density distributions emerge naturally from the central limit theorem, when numerous independent shocks induce multiplicative and random density perturbations \citep{Vaz01}. The disadvantage of these theoretical predictions is that they do not account for many factors that confound observations; opacity, chemical or temperature variations, and spatial filtering affect measured cloud properties, making interpretation more difficult.

I believe we can better understand molecular clouds with improvements to the way we interpret observations and simulations of these objects. I have focused on two strategies to drive this improvement:

\begin{itemize}
\item{\textbf{Structure Identification}} Structure identification is a crucial early step in most molecular cloud analysis, and early blob-segmentation techniques do not adequately describe modern cloud datasets. Part of my thesis has focused on further developing hierarchical cloud segmentation methods \citep{Rosolowsky08}. These techniques better-describe molecular clouds, which are naturally hierarchical. I have also developed techniques to identify ISM structures using machine learning. Machine learning has not been applied to the task of ISM identification previously. However, I have found that it is an effective approach to identify features based on otherwise difficult-to-quantify morphological patterns that humans are adept at identifying.
\item{\textbf{Synthetic Observation Analysis}} Observations are not clean measurements of the physical properties of molecular clouds. Instead, derived cloud properties are affected in complicated ways by radiative transfer, superposition, spatial abundance and temperature variations, noise, and spatial filtering. The only practical way to realistically account for these confounding factors is to study synthetic observations of molecular cloud simulations, where the true state of the cloud is known and can be compared to observations \citep{Goodman11}.
\end{itemize}

\section{Outline}
The chapters that follow represent five specific research projects I have pursued during my graduate work.

Chapter 2 (first published as \citealt{Beaumont10}) was my first foray into molecular cloud research, and investigates how Carbon Monoxide observations of stellar-driven cavities in molecular clouds constrain the three-dimensional geometry of these clouds. In particular, we found that many of these ``bubbles'' appear ringlike in sub-mm emission -- that is, the fronts and backs of many bubbles are missing. The implication of a ring-like geometry is that many star-forming molecular clouds are sheet-like, where the thin dimension of the cloud is $\lesssim$ 5pc thick. This is a relevant constraint for theories of molecular cloud formation, and it also implies that bubbles are relatively inefficient at transferring energy back into molecular clouds; most of the energy flux escapes out of the thin axis of the cloud, and never interacts with molecular material.

Chapter 3 (first published as \citealt{Beaumont11}) represents my first investigation into leveraging machine learning to perform more sophisticated structure identification. This chapter focuses on a PPV dataset with two distinct yet overlapping structures -- the M17 molecular cloud, and the G16.05-0.57 supernova remnant. The two objects are blended in the dataset, but have different morphologies; M17 is spatially diffuse with $\sim 1$ km s$^{-1}$ linewidths, while the supernova remnant is comprised of several thin filaments with 5-10 km s$^{-1}$ features. The Support Vector Machine algorithm is able to infer these morphological differences when trained using a representative set of pixels belonging to each object. We use this classifier to associate each pixel with either M17 or the supernova remnant -- a task that would be infeasible by hand. By performing a pixel-level decomposition, we were able to isolate emission from each object, and derive better dynamical measurements of the mass and momentum of the supernova remnant. This work demonstrates the utility of machine learning algorithms to perform complex structure identification tasks in ISM studies.

Chapter 4 (first published as \citealt{Beaumont12}) is a brief and self-contained paper that attempts to unify the interpretation of two common cloud statistics: the mass-size relationship of cloud substructures, and the cloud column density distribution. The former is an example of a local cloud description involving substructure identification, while the latter is a global description. These two statistics are often treated independently, but they are clearly related; a cloud with more material at high column density will possess substructures with greater mass at a given size. Furthermore, the mass-size relationship has been interpreted problematically in the past. Since \cite{Larson81} first reported that cloud masses scale with size as $M \propto R^2$, people have made the imprecise claim that clouds have constant surface density $\Sigma = \frac{M}{R^2}$. This is patently untrue, since column density maps of molecular clouds show a wide range of surface densities. Chapter 4 rephrases the mass-size relationship in terms of the column density distribution and its spatial variation within and among clouds. This rephrasing better illuminates how global cloud statistics like the column density distribution relate to local statistics of substructures.

Chapter 5 (in press as \citealt{Beaumont13a}) tackles the difficult problem of quantifying how projection effects influence measurements of cloud properties. Because clouds are always observed in projection, measurements of cloud intensity in PP or PPV relates only indirectly to information about 3D (Position-Position-Position, or PPP) cloud structure. A very common -- and largely untested -- assumption among molecular cloud researchers is that cloud substructures along the same line of sight move at different velocities, such that structures identified in PPV cubes correspond to density structures in 3D. In Chapter 5 I test that assumption, using synthetic observations of molecular cloud simulations. I present a new technique to measure the detailed correspondence between PPV and PPP substructures in a cloud, which allows us to identify which regions of a cloud are most heavily affected by superposition. We also use this comparison to measure how well measurements like substructure mass, velocity dispersion, size, and virial parameter recover the true values. We explore how disabling individual factors like opacity or chemistry affect cloud measurements. We generally find that projection effects introduce a 30-100\% scatter in measured cloud properties. This is particularly problematic for assessing gravitational boundedness, since most clouds are observed to be within a factor of two of equipartition between kinetic and gravitational potential energy.

Finally, Chapter 6 (submitted as \citealt{Beaumont13b}) returns to the study of bubbles first discussed in Chapter 2, as well as the machine learning techniques in Chapter 3. To date, the majority of known interstellar bubbles in our Galaxy have been identified by the Milky Way Project -- a citizen scientist initiative where the public identifies bubbles in images from the \textit{Spitzer Space Telescope} \citep{Simpson12}. However, citizen science searches will need to be augmented by algorithmic searches in order to handle the next generation of large astrophysical datasets. We developed a technique called Brut to train a machine learning algorithm to detect bubbles, based on the Milky Way Project catalog. Brut is successfully able to detect bubbles based on their approximately circular morphologies. More importantly, it provides a confidence measure for each bubble candidate that can be calibrated against the classifications of expert astronomers. This allows us to reassess the objects in the Milky Way Catalog, to identify probable non-bubble interlopers. We find that 10-30\% of the catalog contains objects that an expert astronomer would be unlikely to classify as a bubble. Brut was able to identify several biases in the Milky Way Project, including an over-identification of false bubbles near giant HII regions and away from the Galactic midplane. More broadly, Brut demonstrates the synergies that exist between citizen science and machine learning approaches to structure identification, and suggests a path forward for analyzing increasingly large datasets of the interstellar medium.



\begin{thebibliography}{16}
\expandafter\ifx\csname natexlab\endcsname\relax\def\natexlab#1{#1}\fi

\bibitem[Barnard(1919)]{Barnard19} Barnard, E.~E.\ 1919, \apj,
49, 1

\bibitem[Bates
\& Spitzer(1951)]{Bates51} Bates, D.~R., \& Spitzer, L., Jr.\ 1951, \apj, 113, 441

\bibitem[Beaumont
\& Williams(2010)]{Beaumont10} Beaumont, C.~N., \& Williams, J.~P.\ 2010, \apj, 709, 791


\bibitem[Beaumont et al.(2011)]{Beaumont11} Beaumont, C.~N.,
Williams, J.~P., \& Goodman, A.~A.\ 2011, \apj, 741, 14


\bibitem[Beaumont et al.(2012)]{Beaumont12} Beaumont, C.~N.,
Goodman, A.~A., Alves, J.~F., et al.\ 2012, \mnras, 423, 2579

\bibitem[Beaumont et al.(2013a)]{Beaumont13a} Beaumont, C.~N.,
Offner, S.~S.~R., Shetty, R., Glover, S.~C.~O.,
\& Goodman, A.~A.\ 2013, arXiv:1310.1929

\bibitem[Beaumont et al.(2013b)]{Beaumont13b} Beaumont, C.~N.,
Goodman, A.~A., Williams, J.~P., Kendrew, S., Simpson, R.\ 2013, ApJ submitted

\bibitem[Elmegreen
\& Scalo(2004)]{Elmegreen04} Elmegreen, B.~G., \& Scalo, J.\ 2004, \araa, 42, 211

\bibitem[Hartmann(1904)]{Hartmann04} Hartmann, J.\ 1904, \apj, 20,  338

\bibitem[Herzberg(1951)]{Herzberg51} Herzberg, G.\ 1951, \aj, 56, 118

\bibitem[Goodman(2011)]{Goodman11} Goodman, A.~A.\ 2011,  Computational Star Formation, 270, 511

\bibitem[Goldreich
\& Kwan(1974)]{Goldreich74} Goldreich, P., \& Kwan, J.\ 1974, \apj, 189, 441

\bibitem[Gundermann et al.(1965)]{Gundermann65} Gundermann, E.~J.,
Goldstein, S.~J., Jr., \& Lilley, A.~E.\ 1965, \aj, 70, 321

\bibitem[Larson(1981)]{Larson81} Larson, R.~B.\ 1981, \mnras,
194, 809

\bibitem[Palmer
\& Zuckerman(1967)]{Palmer67} Palmer, P., \& Zuckerman, B.\ 1967, \apj, 148, 727

\bibitem[Pineda et al.(2009)]{Pineda09} Pineda, J.~E.,
Rosolowsky, E.~W., \& Goodman, A.~A.\ 2009, \apjl, 699, L134

\bibitem[V{\'a}zquez-Semadeni et al.(2007)]{Vaz07}
V{\'a}zquez-Semadeni, E., G{\'o}mez, G.~C., Jappsen, A.~K., et al.\ 2007,
\apj, 657, 870

\bibitem[V{\'a}zquez-Semadeni
\& Garc{\'{\i}}a(2001)]{Vaz01} V{\'a}zquez-Semadeni, E., \& Garc{\'{\i}}a, N.\ 2001, \apj, 557, 727

\bibitem[Rosolowsky et al.(2008)]{Rosolowsky08} Rosolowsky, E.~W.,
Pineda, J.~E., Kauffmann, J., \& Goodman, A.~A.\ 2008, \apj, 679, 1338

\bibitem[Simpson et al.(2012)]{Simpson12} Simpson, R.~J., Povich,
M.~S., Kendrew, S., et al.\ 2012, \mnras, 424, 2442

\bibitem[Stutzki
\& Guesten(1990)]{Stutzki90} Stutzki, J., \& Guesten, R.\ 1990, \apj, 356, 513

\bibitem[Williams et al.(1994)]{Williams94} Williams, J.~P., de
Geus, E.~J., \& Blitz, L.\ 1994, \apj, 428, 693

\bibitem[Wilson et al.(1970)]{Wilson70} Wilson, R.~W., Jefferts,
K.~B., \& Penzias, A.~A.\ 1970, \apjl, 161, L43

\end{thebibliography}
