\documentclass[]{uhthesis}

\usepackage{amssymb,amsmath,gensymb}
%\usepackage{chapterbib}
\usepackage{csquotes}
%\usepackage{hyperref}
%\usepackage{ifxetex,ifluatex}
\usepackage{natbib}
\usepackage{uhaas}
\usepackage{amsmath}

\usepackage{chapterfolder}
% and we re-write includegraphics

\usepackage{graphicx}
\usepackage{epstopdf}
 \DeclareGraphicsExtensions{.pdf,.eps,.png}
 \let\includegraphicsWithoutCF\includegraphics
\renewcommand{\includegraphics}[2][]{\includegraphicsWithoutCF[#1]{\cfcurrentfolder#2}}
 

\newcommand{\kms}[0]{km\,s$^{-1}$}
\newcommand{\glimpse}[0]{{\sc glimpse} }
\newcommand{\vla}[0]{{\sc vla} }
\newcommand{\snr}[0]{G16.05-0.57}
\newcommand{\magpis}[0]{{\sc magpis} }
\newcommand{\citepy}[1]{\citeauthor{#1} (\citeyear{#1})}
\newcommand{\hii}[0]{H{\sc ii} }
\newcommand{\jcmt}[0]{{\sc jcmt} }
\newcommand{\rhopdf}[0]{$\rho-$PDF}
\newcommand{\npdf}[0]{$\mathcal{N}$-PDF}
\newcommand{\coa}[0]{$^{12}$CO (J=1-0)}
\newcommand{\cob}[0]{$^{12}$CO (J=3-2)}
\newcommand{\coc}[0]{$^{13}$CO (J=1-0)}
\newcommand{\um}[0]{$\mu$m}
\DeclareMathOperator*{\argmin}{arg\,min}
\DeclareMathOperator*{\argmax}{arg\,max}

\begin{document}
\frontmatter

\title{Morphological Diagnostics of Star Formation in Molecular Clouds}
\author{Christopher Norris Beaumont}
\date{October 12, 2013}
\chairperson{Jonathan Williams}
\memberA{Alyssa Goodman}
\memberB{Bo Reipurth}
\memberC{Istvan Szapudi}
\memberD{Kim Binsted}
\maketitle
\makesig
\makecopyright{2013}
{Christopher Beaumont}

\makeacknowledgements
blash

\makeabstract
Molecular clouds are the birth sites of all star formation in the present-day universe. They represent the initial conditions of star formation, and are the primary medium by which stars transfer energy, momentum, and turbulence back to parsec-scales. Yet, the physical evolution of molecular clouds remains poorly understood. This is not due to a lack of observational data, nor is it due to an inability to simulate the conditions inside molecular clouds. Instead, the physics and morphology of the interstellar medium is sufficiently complex that interpreting molecular cloud data is very difficult. This dissertation mitigates this problem, by developing more sophisticated ways to interpret morphological information in molecular cloud observations and simulations. In particular, I have focused on leveraging machine learning techniques to identify physically meaningful substructures in the interstellar medium, as well as techniques to inter-compare molecular cloud simulations to observations.  These contributions make it easier to understand the interplay between molecular clouds and star formation. Specific contributions include: new insight about the sheet-like geometry of molecular clouds based on observations of stellar bubbles; a new algorithm to disambiguate overlapping yet morphologically distinct cloud structures in spectral data cubes; a new perspective on the relationship between molecular cloud column density distributions and the sizes of cloud substructures; a quantitative analysis of how projection effects effect measurements of cloud properties; and a automatically generated, statistically-characterized catalog of bubbles identified from their infrared morphologies.

\tableofcontents
\listoftables
\listoffigures

\mainmatter

\cfchapter{Introduction}{ch1_intro}{intro.tex}
\cfchapter{Molecular Rings around Interstellar Bubbles and the Thickness of Star-Forming Clouds}{ch2_bubbles}{ms.tex}
\cfchapter{Classifying structures in the ISM with Support Vector Machines: the G16.05-0.57 supernova remnant}{ch3_snr}{ms.tex}
\cfchapter{A Simple Perspective on the Mass-Area Relationship in Molecular Clouds}{ch4_larson}{ms.tex}
\cfchapter{Quantifying Observational Projection Effects Using Molecular Cloud Simulations}{ch5_ppv}{ms.tex}
\cfchapter{The Milky Way Project: Leveraging Citizen Science and Machine Learning to Detect Interstellar Bubbles}{ch6_brut}{ms.tex}

\end{document}